
\documentclass[10pt,twoside,slovak,a4paper]{article}
\usepackage[slovak]{babel}
\usepackage[IL2]{fontenc} 
\usepackage[utf8]{inputenc}
\usepackage{graphicx}
\usepackage{url} 
\usepackage{hyperref} 
\usepackage{cite}
\pagestyle{headings}
\title{Hodnotenie skutočnej a vnímanej námahy poskytovanej hrami virtuálnej reality\thanks{Semestrálny projekt v predmete Metódy inžinierskej práce, ak. rok 2022/23, vedenie: Ivan Kapústik}} 
\author{Marek Biensky\\[2pt]
	{\small Slovenská technická univerzita v Bratislave}\\
	{\small Fakulta informatiky a informačných technológií}\\
	{\small \texttt{xbiensky@stuba.sk}}
	}
\date{\small 25. októbra 2022} 


\begin{document}

\maketitle

\begin{abstract}
\ldots
\end{abstract}



\section{Úvod}
Článok sa zaoberá technológiou VR ,ktorá neslúži len na zábavu ,ale dokáže naozaj pomôcť.
Hry virtuálnej reality preukázali potenciál urobiť cvičenie zábavným a pútavým.
Dokonca sa používa aj pri rehabilitáciach pomocou technologie kinect od microsoftu,ktorá sa začala predávať s konzolami xbox.
Pomáha aj pri liečení fóbie z úrazov ako je napríklad autonehoda.
 Všetko je podrobne vysvetléné v časti ~\ref{VR sport}.
Dôležité súvislosti sú uvedené v častiach~\ref{VR rehabilitácie} a~\ref{Pre zdravie}.
Záverečné poznámky prináša časť~\ref{zaver}.


\section{VR sport} \label{VR sport}

cvičenie a udržiavanie kondície, ktoré pomáha starším ľuďom precvičiť si myseľ pri hraní hry
telesného cvičenia, nazýva sa to dual-task a je to aktivita, ktorá sa považuje za užitočnú pri zlepšovaní
fyzické a kognitívne funkcie.
V súčasnosti existujú programy, ktoré podporujú zdravie starších ľudí týmto spôsobom ako napríklad kontrola dýchania pohybu a daľsieho. množstvo platforiem
hry ako Wii, PlayStation, Wii Balance, Xbox, s ktorými môžete použivať pohybové sústavy ako kinect.
Napíšem niečo aj o Kinecte
Kinect je kamera so snímačom hĺbky od spoločnosti Microsoft, ktorá nám poskytuje informácie o hĺbke, farbe a kostrách ľudí
ktorí stoja pred ním. Tieto informácie sa používajú na výpočet uhlov a pozícií každej osoby.

%Z obr.~\ref{f:rozhod} je všetko jasné. 

\begin{figure*}[tbh]
\centering
%\includegraphics[scale=1.0]{diagram.pdf}
Aj text môže byť prezentovaný ako obrázok. Stane sa z neho označný plávajúci objekt. Po vytvorení diagramu zrušte znak \texttt{\%} pred príkazom \verb|\includegraphics| označte tento riadok ako komentár (tiež pomocou znaku \texttt{\%}).
\caption{Rozhodujúci argument.}
\label{f:rozhod}
\end{figure*}



\section{VR rehabilitácie} \label{VR rehabilitácie}

Základným problémom je teda\ldots{} Najprv sa pozrieme na nejaké vysvetlenie (časť~\ref{VR rehabilitácie}), a potom na ešte nejaké (časť~\ref{VR rehabilitácie}).\footnote{Niekedy môžete potrebovať aj poznámku pod čiarou.}

Môže sa zdať, že problém vlastne nejestvuje\cite{Coplien:MPD}, ale bolo dokázané, že to tak nie je~\cite{Czarnecki:Staged, Czarnecki:Progress}. Napriek tomu, aj dnes na webe narazíme na všelijaké pochybné názory\cite{PLP-Framework}. Dôležité veci možno \emph{zdôrazniť kurzívou}.


\subsection{Pre zdravie} \label{Pre zdravie}

Niekedy treba uviesť zoznam:

\begin{itemize}
\item jedna vec
\item druhá vec
	\begin{itemize}
	\item x
	\item y
	\end{itemize}
\end{itemize}

Ten istý zoznam, len číslovaný:

\begin{enumerate}
\item jedna vec
\item druhá vec
	\begin{enumerate}
	\item x
	\item y
	\end{enumerate}
\end{enumerate}


\subsection{Pokracovanie} \label{Pre zdravie}

\paragraph{Veľmi dôležitá poznámka.}
Niekedy je potrebné nadpisom označiť odsek. Text pokračuje hneď za nadpisom.



\section{Dôležitá časť} \label{dolezita}




\section{Ešte dôležitejšia časť} \label{dolezitejsia}




\section{Záver} \label{zaver} % prípadne iný variant názvu



%\acknowledgement{Ak niekomu chcete poďakovať\ldots}


% týmto sa generuje zoznam literatúry z obsahu súboru literatura.bib podľa toho, na čo sa v článku odkazujete
\bibliography{literatura}
\bibliographystyle{plain} % prípadne alpha, abbrv alebo hociktorý iný
\end{document}
