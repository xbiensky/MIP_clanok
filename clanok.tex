
\documentclass[10pt,oneside,slovak,a4paper]{article}
\usepackage[slovak]{babel}
\usepackage[IL2]{fontenc} 
\usepackage[utf8]{inputenc}
\usepackage{graphicx}
\usepackage{url} 
\usepackage{hyperref} 
\usepackage{cite}
\pagestyle{headings}
\title{Hodnotenie skutočnej a vnímanej námahy poskytovanej hrami virtuálnej reality\thanks{Semestrálny projekt v predmete Metódy inžinierskej práce, ak. rok 2022/23, vedenie: Ivan Kapústik}} 
\author{Marek Biensky\\[2pt]
	{\small Slovenská technická univerzita v Bratislave}\\
	{\small Fakulta informatiky a informačných technológií}\\
	{\small \texttt{xbiensky@stuba.sk}}
	}
\date{\small 25. októbra 2022} 

\begin{document}

\maketitle

\begin{abstract}
 \ldots V posledných rokoch si medzi mladými používateľmi získala popularitu pohlcujúca virtuálna realita (VR) ako nová technológia pre zábavné hry.Obrazom toho sa možete dočítať v uvedenom článku nižšie prostredníctvom akých hier a akých technologií.
\end{abstract}


\section{Úvod}
Článok sa zaoberá technológiou VR ,ktorá neslúži len na zábavu ,ale dokáže naozaj pomôcť.
Hry virtuálnej reality preukázali potenciál urobiť cvičenie zábavným a pútavým.
Dokonca sa používa aj pri rehabilitáciach pomocou technologie kinect od microsoftu,ktorá sa začala predávať s konzolami xbox.
Pomáha aj pri liečení fóbie z úrazov ako je napríklad autonehoda.

% Všetko je podrobne vysvetléné v časti ~\ref{VR sport}.
%Dôležité súvislosti sú uvedené v častiach~\ref{VR rehabilitácie} a~\ref{Pre zdravie}.
%Záverečné poznámky prináša časť~\ref{zaver}.


\section{VR sportyt} \label{VR sport}

cvičenie a udržiavanie kondície, ktoré pomáha starším ľuďom precvičiť si myseľ pri hraní hry
telesného cvičenia, nazýva sa to dual-task a je to aktivita, ktorá sa považuje za užitočnú pri zlepšovaní
fyzické a kognitívne funkcie.
V súčasnosti existujú programy, ktoré podporujú zdravie starších ľudí týmto spôsobom ako napríklad kontrola dýchania pohybu a daľsieho. množstvo platforiem
hry ako Wii, PlayStation, Wii Balance, Xbox, s ktorými môžete použivať pohybové sústavy ako kinect.
Napíšem niečo aj o Kinecte
Kinect je kamera so snímačom hĺbky od spoločnosti Microsoft, ktorá nám poskytuje informácie o hĺbke, farbe a kostrách ľudí
ktorí stoja pred ním. Tieto informácie sa používajú na výpočet uhlov a pozícií každej osoby.

%Z obr.~\ref{f:rozhod} je všetko jasné. 

%\begin{figure*}[tbh]
%\centering
%\includegraphics[scale=1.0]{diagram.pdf}
%Aj text môže byť prezentovaný ako obrázok. Stane sa z neho označný plávajúci objekt. Po vytvorení diagramu zrušte znak \texttt{\%} pred príkazom \verb|\includegraphics| označte tento riadok ako komentár (tiež pomocou znaku \texttt{\%}).
%\caption{Rozhodujúci argument.}
%\label{f:rozhod}
%\end{figure*}

\section{VR cviky} \label{VR rehabilitácie}
Napríklad
silové hry
V silových hrách musí pacient cvičiť niektoré cviky, ako keby bol v telocvični. Platforma vysvetľuje
aké cviky má robiť, koľko času má na odpočinok a akú váhu by mal prijať.
Scéna sa nachádza napríklad na krásnej pláži, kde pacient musí pohnúť avatarom vlastným telom a tam je
tiež ďalší avatar, ktorý naznačuje pohyby, ktoré by mal robiť.
Systém je schopný zistiť, či 
pohyb bol vykonaný správne. Ak sa pacientovi nepodarí vykonať cvičenie v danom časovom období,
systém bude vedieť a prejde na ďalšie cvičenie. Všetky údaje o relácii, ako je čas v každom cvičení alebo čas
odpočinok sa uloží, aby mohol byť jeho terapeut informovaný.
Pozname taktiez vodné rehabilitácie ktoré odporúčaným rehabilitačným prístupom pri mnohých zraneniach a postihnutiach, pretože udržiava pacientov v pohode, vedie k nízkej záťaži kĺbov pacientov a ponúka dodatočný odpor na zlepšenie účinnosti cvičenia. Rastúce využívanie vodnej rehabilitácie a výhody virtuálnej rehabilitácie na zemi zvyšujú potrebu použiteľných a dostupných systémov virtuálnej reality (VR), ktoré fungujú pod vodou. Jedným z prvých príkladov prispôsobenia systému VR alebo systému rozšírenej reality (AR) pod vodou bol Morales et al [3]podvodný AR systém, v ktorom mali používatelia k dispozícii vodotesný priehľadný displej na hlave, ktorý im umožňoval plávať v skutočnom bazéne s virtuálnymi rybami alebo vizualizovať komerčné montážne potápačské úlohy.
Shark Punch je nová podvodná hra pre virtuálnu realitu (VR), v ktorej hráči musia bojovať o život v skutočnom podmorskom prostredí proti virtuálnemu veľkému bielemu žralokovi. Z rehabilitačného hľadiska sa táto hra zameriava na zlepšenie pohyblivosti kĺbov a svalovej sily flexorov a extenzorov prostredníctvom úderových cvičení. Konkrétne, žralok najprv obíde hráča a potom zúrivo zaútočí a snaží sa hráča uhryznúť. Po zahryznutí sa stiahne späť, aby obchádzal hráča z diaľky. Uhryznutiu je možné zabrániť iba vtedy, ak používateľ zasadí skutočný úder do nosa virtuálneho žraloka 
\cite{PEDRAZAHUESO2015161}
\cite{doi:10.1089/109493103322011641}
%Základným problémom je teda\ldots{} Najprv sa pozrieme na nejaké vysvetlenie (časť~\ref{VR rehabilitácie}), a potom na ešte nejaké (časť~\ref{VR rehabilitácie}).\footnote{Niekedy môžete potrebovať aj poznámku pod čiarou.}

%Môže sa zdať, že problém vlastne nejestvuje\cite{Coplien:MPD}, ale bolo dokázané, že to tak nie je~\cite{Czarnecki:Staged, Czarnecki:Progress}. Napriek tomu, aj dnes na webe narazíme na všelijaké pochybné názory\cite{PLP-Framework}. Dôležité veci možno \emph{zdôrazniť kurzívou}.

%\subsection{Pre zdravie} \label{Pre zdravie}

\section{Záver} \label{zaver} % prípadne iný variant názvu
Určite na záver treba pripomenúť to že  hry virtuálnej reality začali získavať na popularite u masy ľudí, rastúca závislosť na týchto hrách bola celkom typická.sociálne médiá, môže viesť k závislosti a hranie vo VR nie je výnimkou. Preto je potrebné prijať vhodné opatrenia na zvládnutie takýchto závislostí.Čo si myslím že ich je mnoho a je lahké sa k nim dočítať
\cite{8649547}



%\acknowledgement{Ak niekomu chcete poďakovať\ldots}


% týmto sa generuje zoznam literatúry z obsahu súboru literatura.bib podľa toho, na čo sa v článku odkazujete
\bibliography{name.bib}
\bibliographystyle{plain} % prípadne alpha, abbrv alebo hociktorý iný
\end{document}

